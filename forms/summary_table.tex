%==================================================
\phantom{x}	\vspace{1cm}
\begin{tabular}{r|r|rrrrr}
	\multicolumn{7}{r}{(金額単位:千円)}\\
	\hline
	年度 & 年度合計 & 設備備品 & 消耗品 &旅費 & 謝金等 & その他\\
	\hline
	\1年目西暦(H\1年目) & 
		\NumC{KLAnnualSum1} &
		\NumC{KLequipments1} & \NumC{KLexpendables1} & 
		\NumC{KLtravel1} & \NumC{KLgratitude1} & \NumC{KLmisc1}\\
	\hline
	\2年目西暦(H\2年目) & 
		\NumC{KLAnnualSum2} &
		\NumC{KLequipments2} & \NumC{KLexpendables2} & 
		\NumC{KLtravel2} & \NumC{KLgratitude2} & \NumC{KLmisc2}\\
	\hline
	\3年目西暦(H\3年目) & 
		\NumC{KLAnnualSum3} &
		\NumC{KLequipments3} & \NumC{KLexpendables3} & 
		\NumC{KLtravel3} & \NumC{KLgratitude3} & \NumC{KLmisc3}\\
	\hline
	\4年目西暦(H\4年目) & 
		\NumC{KLAnnualSum4} &
		\NumC{KLequipments4} & \NumC{KLexpendables4} & 
		\NumC{KLtravel4} & \NumC{KLgratitude4} & \NumC{KLmisc4}\\
	\hline
	\5年目西暦(H\5年目) & 
		\NumC{KLAnnualSum5} &
		\NumC{KLequipments5} & \NumC{KLexpendables5} & 
		\NumC{KLtravel5} & \NumC{KLgratitude5} & \NumC{KLmisc5}\\
	\hline
	\6年目西暦(H\6年目) & 
		\NumC{KLAnnualSum6} &
		\NumC{KLequipments6} & \NumC{KLexpendables6} & 
		\NumC{KLtravel6} & \NumC{KLgratitude6} & \NumC{KLmisc6}\\
	\hline
%%	\7年目西暦(H\7年目) & 
%%		\NumC{KLAnnualSum7} &
%%		\NumC{KLequipments7} & \NumC{KLexpendables7} & 
%%		\NumC{KLtravel7} & \NumC{KLgratitude7} & \NumC{KLmisc7}\\
%%	\hline
	\hline
	合計 &
		\NumC{KLAnnualSum0} &
		\NumC{KLequipments0} & \NumC{KLexpendables0} & 
		\NumC{KLtravel0} & \NumC{KLgratitude0} & \NumC{KLmisc0}\\
	\hline
		\setcounter{KLGrandTotalValue}{0}
		\addtocounter{KLGrandTotalValue}{\value{KLequipments0}}
		\addtocounter{KLGrandTotalValue}{\value{KLexpendables0}}
		\addtocounter{KLGrandTotalValue}{\value{KLtravel0}}
		\addtocounter{KLGrandTotalValue}{\value{KLgratitude0}}
		\addtocounter{KLGrandTotalValue}{\value{KLmisc0}}
	各品目の合計 & \NumC{KLGrandTotalValue}
			& \multicolumn{5}{l}{
				\ifthenelse{\value{KLAnnualSum0} = \value{KLGrandTotalValue}}{
					
				}{
					ERROR!! 
				}
			}\\
	\hline

\end{tabular}

\vspace{1cm}
\underline{チェックリスト}
\begin{enumerate}
	\item \LaTeX のソースの中で、各品目の金額が必ず\textbackslash KLItemCost 、
		\textbackslash KLItemNumUnitCost、\\
		\textbackslash KLItemNumUnitCostTwo などを用いて書かれていることを確かめてください。
		これらのマクロを使って書かれた金額の合計が、一番下の段の「各品目の合計」です。

%	\item 各年度、各項目(設備備品、消耗品、..)ごとの金額は、
%		\LaTeX の表の中に書かれた、年度ごとの小計を表す \textbackslash KLSum を用いて
%		計算されています。
%		必ず、表の中では \textbackslash KLSum を用いてください。
%		
%	\item 応募様式の中の表の「年度ごとの」小計と、上の表は一致していますか?
%		(\textbackslash KLSumが使われていないと、年度がずれます。)
	
	\item 研究種目ごとに、申請予算の上限が定められています。公募要領をよく読んで確かめてください。
	\item この表に現れる金額と、電子申請の際の「応募情報入力」の金額が、全て一致していることを
		確かめてください。
		
	\item まさか、「象の卵」のための項目や金額は、もう残っていませんね??
		
	\item 問題がなければ、 \KLMainFile の初めの付近にある行を\\
				\hspace{2cm}\%\textbackslash setboolean\{BudgetSummary\}\{true\}\\
		のようにコメントアウトし、コンパイルし直して、「応募内容ファイル」を
		作り直してください。電子申請で送れるファイル形式は、「PDF」です。
		PSは受け付けられません。
		
	\item 電子申請で送るファイルのサイズが、\underline{3MB以下}であることを確かめてください。
		もし、3MBを越える場合は、読み込んでいるきれいな図形の解像度を落としてください。
		また、読み込む様式ファイルの形式(eps or pdf)を変えると
		(\textbackslash usePDFform\{true\}のコメントをつける or はずす)、
		最終的にできるファイルの大きさは変わります。
\end{enumerate}

