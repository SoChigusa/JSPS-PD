%\documentclass[11pt,a4paper,uplatex,twoside,dvipdfmx]{ujarticle} 	% for uplatex
\documentclass[11pt,a4paper,twoside,dvipdfmx]{jarticle}		% platex
%==== 科研費LaTeX =============================================
%	2020(H32)年度 PD
%============================================================
% 2008-03-08: Taku Yamanaka (JSPS Research Center for Science Systems / Osaka Univ.)
% 2008-03-10: Taku; Fixed a bug which was missing p.7.
% 2009-03-04: K.S.: Revised for JFY2010.
% 2010-03-04: Taku: Revised for JFY2011.
% 2011-03-20: Taku: Revised for JFY2012.
% 2012-02-25: Taku: Revised for JFY2013.
% 2013-03-14: Taku: Revised for JFY2014.
% 2014-03-02: Taku: Revised for JFY2015.
% 2015-02-23: Taku: Revised for JFY2016.
% 2016-02-26: Taku: Revised for JFY2017.
%============================================================
\input{forms/form00_header}
% user01_header
%=== 様式のファイルの形式の指定 =================
%   PDFではなく、eps の様式を読み込む場合は、次の行の頭に「%」をつけてください。
\setboolean{usePDFform}{true}
%===================================
\input{forms/form01_header}

% user02_header
%=== 予算の表の印刷 =====================
% 予算の集計の表を出すためには、次の行の頭の%を消してください。
%\setboolean{BudgetSummary}{true}
%=================================

%=== For English, uncomment the next line to left-justify inside table columns.
%\renewcommand{\KLCLLang}{\KLCL}

% === 一部のページだけタイプセット ==============
% New in 2009 fall version!
% 選んだページだけタイプセットするには、次の例の頭の%を消し、並べてください。
% 複数のページを選ぶこともできます。
% 提出前には、必ず全てコメントアウト(頭に%をつける)してください。
%ーーーーーーーーーーーーーーーーーーーーーーーーーーーーーーーーー
%\KLTypesetPage{1}			% p.1 (or p.1を含む連続したページ),
%\KLTypesetPage{3}			% p.3 (or p.3を含む連続したページ),
%\KLTypesetPagesInRange{5}{6}	% p.5 ~ p.6,
%\KLTypesetPagesInRange{8}{10}	% and p.8 ~ p.10
%=================================

% ===== my favorite packages ====================================
% ここに、自分の使いたいパッケージを宣言して下さい。
\usepackage{wrapfig}
% \usepackage{amssymb}
%\usepackage{mb}
% \usepackage{color} % でも科研費の書類はグレースケールで印刷されます
%\DeclareGraphicsRule{.tif}{png}{.png}{`convert #1 `dirname #1`/`basename #1 .tif`.png}
%==========================================================

\newcommand{\KLShouKeiLine}[1]{\cline{#1}}
%もし、小計の上の線を取れと事務に言われたら、
%「そのようなことは、記入要項に書かれていないし、学振はそのようなことは気にしていない。」と
% 突っぱねる。
% それでもなお消せと理不尽なことを言われたら、次の行の 最初の「%」を消す。	
%\renewcommand{\KLShouKeiLine}[1]{}

\newcommand{\KLBudgetTableFontSize}{small}	% 予算の表のフォントの大きさ: small, footnotesize
\newcommand{\KLFundsTableFontSize}{small}	%応募中、受入れ予定の研究費のフォントの大きさ:normalsize, small, footnotesize

% ===== my personal definitions ==================================
% ここに、自分のよく使う記号などを定義して下さい。
\newcommand{\klpionn}{K_L \to \pi^0 \nu \overline{\nu}}
\newcommand{\kppipnn}{K^+ \to \pi^+ \nu \overline{\nu}}


\input{forms/hook3} % for future maintenance
% ===== Global definitions for the PD form ======================
% 基本情報
%
%------ 研究課題名  -------------------------------------------
\newcommand{\研究課題名}{象の卵}

%----- 研究機関名と研究代表者の氏名-----------------------
\newcommand{\研究機関名}{逢坂大学}
\newcommand{\申請者氏名}{湯川秀樹}
\newcommand{\研究代表者氏名}{\申請者氏名}

%---- 研究期間の最終年度 ----------------
\newcommand{\研究期間の最終元号年度}{34}	%平成で、半角数字のみ
%=========================================================
% ===== Global year-dependent definitions for the Kakenhi form ===========
% 基本情報
\newcommand{\研究開始年度}{2020}
\newcommand{\研究開始元号年度}{32}	%平成

\newcommand{\1年目西暦}{2020}
\newcommand{\2年目西暦}{2021}
\newcommand{\3年目西暦}{2022}
\newcommand{\4年目西暦}{2023}
\newcommand{\5年目西暦}{2024}
\newcommand{\6年目西暦}{2025}

\newcommand{\1年目}{32}
\newcommand{\2年目}{33}
\newcommand{\3年目}{34}
\newcommand{\4年目}{35}
\newcommand{\5年目}{36}
\newcommand{\6年目}{37}

\newcommand{\1年目J}{32}
\newcommand{\2年目J}{33}
\newcommand{\3年目J}{34}
\newcommand{\4年目J}{35}
\newcommand{\5年目J}{36}
\newcommand{\6年目J}{37}


	% 
\input{forms/form03_header}
%=================================================================
% form05_pd_header.tex
%	for the 2007(H19) Japanese Fiscal Year
%	2006-10-01 : Taku Yamanaka (Osaka Univ.)
%			Switched to the new development system using a "mother file".
%	2007-08-08: Taku
%			Switched to a new method using "picture" environment.
%	2008-03-08: Taku
%			Readjusted parameters for the new 2008 form.
%	2009-09-04: Taku
%			Introduced form03_header and form07_header to automatically calculate margins and
%			other miscellaneous coordinate parameters.
%=================================================================

% ===== Global definitions for the Kakenhi form ======================
% 基本情報
\newcommand{\研究種目}{PD}
\newcommand{\研究種目後半}{}
\ifthenelse{\isundefined{\研究種別}}{
	\newcommand{\研究種別}{}
}{}%
\newcommand{\KLMainFile}{pd.tex}
\newcommand{\KLForms}{pd_forms}
\newcommand{\KLYoshiki}{pd}

% 奇数ページの下に記入される情報
\newcommand{\klbyYup}{}
\newcommand{\klbyYdown}{}
\newcommand{\klbyKikanXleft}{}
\newcommand{\klbyKikanXright}{}
\newcommand{\klbyNameXleft}{}
\newcommand{\klbyNameXright}{}

\newcommand{\KLBottomInfo}[6]{%
	\ifthenelse{\equal{#1}{}}{%
		\renewcommand{\klbyYup}{62}
		\renewcommand{\klbyYdown}{48}
	}{%
		\renewcommand{\klbyYup}{#1}
		\renewcommand{\klbyYdown}{#2}
	}
	
	\ifthenelse{\equal{#3}{}}{%
		\renewcommand{\klbyKikanXleft}{132}
		\renewcommand{\klbyKikanXright}{349}
		\renewcommand{\klbyNameXleft}{425}
		\renewcommand{\klbyNameXright}{550}
	}{%
		\renewcommand{\klbyKikanXleft}{#3}
		\renewcommand{\klbyKikanXright}{#4}
		\renewcommand{\klbyNameXleft}{#5}
		\renewcommand{\klbyNameXright}{#6}
	}
%	\KLTextBox{\klbyKikanXleft}{\klbyYup}{\klbyKikanXright}{\klbyYdown}{}{\研究機関名}%
	\KLTextBox{\klbyNameXleft}{\klbyYup}{\klbyNameXright}{\klbyYdown}{}{\申請者氏名}%
}

%==========================================================
% frame edge positions of multi-page-block
\newcommand{\KLOddMultiPageLeftEdge}{47}
\newcommand{\KLOddMultiPageRightEdge}{549}
\newcommand{\KLEvenMultiPageLeftEdge}{47}
\newcommand{\KLEvenMultiPageRightEdge}{549}

% vertical limits in the first multi-page-block
\newcommand{\KLMultiPageTopEdge}{806}		%lowest top position (except for the 1st page)
\newcommand{\KLMultiPageBottomEdge}{80}	%highest bottom position (except for the last page)

% Modify the edges for single page frames if necessary
\newcommand{\KLOddLeftEdge}{\KLOddMultiPageLeftEdge}
\newcommand{\KLOddRightEdge}{\KLOddMultiPageRightEdge}
\newcommand{\KLEvenLeftEdge}{\KLEvenMultiPageLeftEdge}
\newcommand{\KLEvenRightEdge}{\KLEvenMultiPageRightEdge}

%

\input{forms/form07_header}
%============================================================
%endPrelude

\begin{document}
\input{forms/hook5} % for future maintenance
%============================================================
%     User Inputs
%============================================================

%form: pd_form_03-04.tex ; user: pd_03-04_preparation_etc.tex
%========== PD =========
%===== p. 03-04 現在までの研究状況 =============
\section{現在までの研究状況}
%watermark: w03_past_pd
\newcommand{\現在までの研究状況}{%
%begin  現在までの研究状況===================
	今までは、地球上で最大の生物、シロナガスクジラの卵の研究を進めようとしてきた。
	クジラの卵の場合は、高い水圧に耐える必要があるため、堅固の構造となっているはずであり、
	これが解明されれば、将来、深海潜水艇への応用も効く。
	しかし、シロナガスクジラの生息範囲が広い、海に潜っている時間が長い、
	生息数も減っている、などの原因により、
	卵を見つけることができなかった。
	
	そこで、\underline{地球で}最大の動物から、\underline{地上で}最大の動物に研究対象を変更する。

	\input{blahblah}  % << only for demonstration. Please delete it or comment it out.	
%end  現在までの研究状況 ====================
}

%form: pd_form_05-07.tex ; user: pd_05-07_purpose.tex
%========== PD =========
%===== p. 05-07 これからの研究計画 =============
\section{これからの研究計画}
%watermark: w02_purpose_pd
\subsection{研究の背景}
\newcommand{\研究の背景}{%
%begin  研究の背景===================
	象の卵の研究の背景は...

	\begin{thebibliography}{99}
		\bibitem{teramura} 寺村輝夫、「ぼくは王様 - ぞうのたまごのたまごやき」.
	\end{thebibliography}
%end  研究の背景 ====================
}

\subsection{研究の特色・独創的な点}
\newcommand{\研究の特色と独創的な点}{%
%begin  研究の特色と独創的な点===================
	象の卵の特色と独創的な点は...
%end  研究の特色と独創的な点 ====================
}

\subsection{研究目的}
\newcommand{\研究目的}{%
%begin  研究目的===================
	象の卵の研究の目的は...
	
	\vspace{1cm}
	\begin{thebibliography}{99}
		\bibitem{teramura} 寺村輝夫、「ぼくは王様 - ぞうのたまごのたまごやき」.
	\end{thebibliography}
%end  研究目的 ====================
}

%====================================
%form: pd_form_08.tex ; user: pd_08_plan.tex
%========== PD =========
%===== p. 08 年次計画 =============
\section{年次計画}
\subsection{年次計画}
\newcommand{\採用までの準備}{%
%begin  採用までの準備===================
	採用されるまでは、一所懸命勉強します。
	\vspace{10mm}% adjust the length if necessary
%end  採用までの準備 ====================
}

\newcommand{\年次計画1年目}{%
%begin  年次計画1年目 (figureやtable使用可)===================
	初年度は、まず世界の動物園を巡り、象舍に卵が隠されていないか、探す。
	\vspace{3cm}% adjust the length if necessary
%end  年次計画1年目 (figureやtable使用可) ====================
}

\newcommand{\年次計画2年目}{%
%begin  年次計画2年目 (figureやtable使用可)===================
	2年目はアフリカに行き、空と地上から象の卵を探す。
	アフリカ象は気性が荒いが、サバンナの方がジャングルよりも見通しが効くので、
	インドよりもアフリカを先に探索する。
	\vspace{3cm}% adjust the length if necessary
%end  年次計画2年目 (figureやtable使用可) ====================
}

\newcommand{\年次計画3年目}{%
%begin  年次計画3年目 (figureやtable使用可)===================
	3年目は、インドとタイに行き、ジャングルに隠されている卵を探す。
	ジャングルの場合は空からは探しにくいが、象使いも多く、象の背中に乗って
	象の視点から探索することができる。
	さらに、気だての優しいインド象ならば
	卵の在処を教えてくれる可能性もある。
	\vspace{3cm}% adjust the length if necessary
%end  年次計画3年目 (figureやtable使用可) ====================
}

%    }
%form: pd_form_09.tex ; user: pd_09_rights.tex
%========== PD =========
%===== p. 09 人権の保護及び法令等の遵守への対応 =============
\subsection{人権の保護及び法令等の遵守への対応}
\newcommand{\受け入れ研究室の選定理由}{%
%begin  受け入れ研究室の選定理由===================
	実家から研究室まで徒歩10分、自転車なら3分、電動自転車なら2分58秒、
	全力疾走すれば2分30秒、さらにヘッドスライディングすれば2分29秒で通勤可能だから。
%end  受け入れ研究室の選定理由 ====================
}

\newcommand{\人権の保護及び法令等の遵守への対応}{%
%begin  人権の保護及び法令等の遵守への対応 ===================
	象の卵のES細胞の培養、象のクローンの生成などは行わない。
	象個体を現地から持ち出すことはないので、ワシントン条約ならびに
        生物多様性条約に抵触しない。また、組換え実験は行なわないので、
        カルタヘナ議定書にも抵触しない。
%end  人権の保護及び法令等の遵守への対応 ====================
}

%form: pd_form_10-11.tex ; user: pd_10-11_publications.tex
%========== PD =========
%===== p. 10-11 研究業績 =============
\section{研究業績}
%watermark: w14_pub_pd
% 2008-03-08 Taku
% 2009-03-04 K.S.
% 2010-05-06 Taku
% 2017-03-02 Taku: Added \KLCheckPageLimit and \KLAdvancePages.
\subsection{学術雑誌(紀要・論文集等も含む)に発表した論文及び著書}
\newcommand{\学術雑誌等に発表した論文または著書}{%
%begin  学術雑誌等に発表した論文または著書===================
	
	\begin{enumerate}
		\item[](査読有り)%===========================
		\item \underline{H. Yukawa}$^1$, J. Kara$^2$,
				``Theory of Elephant Eggs'', 
				Phys.\ Rev.\ Lett. {\bf 800}, 800-804 (2005). 
				
		\item F.~Ehrlich, \underline{H. Yukawa}$^1$,
				``You can't Lay an Egg If You're an Elephant'', 
				JofUR\\
				 ({\tt www.universalrejection.org}), {\bf N/A}, N/A (2002).

		\item[](査読なし)%=============================
		\item Kobo Abe$^3$, \underline{H. Yukawa}$^1$, 
				``仔象は死んだ'', 
				安部公房全集, {\bf 26}, 100-200, (2004).
	\end{enumerate}
	他5報
%end  学術雑誌等に発表した論文または著書 ====================
}

\subsection{学術雑誌等又は商業誌における解説・総説}
\newcommand{\学術雑誌等または商業誌における解説や総説}{%
%begin  学術雑誌等または商業誌における解説や総説===================
	\begin{enumerate}
		\item R.~Kipling, \underline{H. Yukawa},
				``The Elephant's Child (象の鼻はなぜ長い)'', 
				Nature, {\bf 999}, 777-779, (2003).
	\end{enumerate}
	他2件
%end  学術雑誌等または商業誌における解説や総説 ====================
}

\subsection{国際会議における発表}
\newcommand{\国際会議における発表}{%
%begin  国際会議における発表===================
	\begin{enumerate}
		\item $\circ$ 湯川秀樹、
			``Theory of Elephant Eggs'', 
			原始殻物理国際会議、
			カラチ、2006年2月

%		\item $\circ$ 湯川秀樹、Jacques-Yves Cousteau,
%			``How to search for whale eggs'',
%			国際海洋探索会議、ハワイ、2003年4月
	\end{enumerate}
	他1件
%end  国際会議における発表 ====================
}

\subsection{国内学会・シンポジウムにおける発表}
\newcommand{\国内学会やシンポジウムにおける発表}{%
%begin  国内学会やシンポジウムにおける発表===================
	\begin{enumerate}
		\item $\circ$ 湯川秀樹、朝永振一郎、
			「ほ乳類の真の意味」、
			ほ乳類学会、
			東京、2003年6月
	\end{enumerate}
	他3件
%end  国内学会やシンポジウムにおける発表 ====================
}

\subsection{特許等}
\newcommand{\特許等}{%
%begin  特許等===================
	\begin{enumerate}
		\item[](公開中)
		\item 800800号、「クジラの卵を用いた深海潜水艇」\underline{湯川秀樹}、2003年4月
%		\item[] (申請中)
%		\item 8000000号、「象の卵を用いた(ひ・み・つ)」、\underline{湯川秀樹}、2007年4月
	\end{enumerate}		
%end  特許等 ====================
}

\subsection{その他の業績}
\newcommand{\その他の業績}{%
%begin  その他の業績===================
		\begin{enumerate}
			\item もうすぐもらえるで賞
		\end{enumerate}
%end  その他の業績 ====================
}

%===========================================================
% hook9 : right before \end{document} ============

%endUserFiles
\input{forms/hook7} % for future maintenance

% pd_forms
%=======================================
\ifthenelse{\boolean{BudgetSummary}\OR\boolean{klTypesetPage0}}{
	\input{forms/coverpage}
}{}

\KLInputIfPageInRangeIsSelected{1}{2}{forms/pd_form_03-04}
\KLInputIfPageInRangeIsSelected{3}{5}{forms/pd_form_05-07}
\KLInputIfSelected{6}{forms/pd_form_08}
\KLInputIfSelected{7}{forms/pd_form_09}
\KLInputIfPageInRangeIsSelected{8}{9}{forms/pd_form_10-11}

%========================================


%endFormatFile

\input{forms/hook9} % for future maintenance
\end{document}
