%\documentclass[11pt,a4paper,uplatex,twoside,dvipdfmx]{ujarticle} 	% for uplatex
\documentclass[11pt,a4paper,twoside,dvipdfmx]{jarticle}		% platex
%==== 科研費LaTeX =============================================
%	2020(H32)年度 PD
%============================================================
% 2008-03-08: Taku Yamanaka (JSPS Research Center for Science Systems / Osaka Univ.)
% 2008-03-10: Taku; Fixed a bug which was missing p.7.
% 2009-03-04: K.S.: Revised for JFY2010.
% 2010-03-04: Taku: Revised for JFY2011.
% 2011-03-20: Taku: Revised for JFY2012.
% 2012-02-25: Taku: Revised for JFY2013.
% 2013-03-14: Taku: Revised for JFY2014.
% 2014-03-02: Taku: Revised for JFY2015.
% 2015-02-23: Taku: Revised for JFY2016.
% 2016-02-26: Taku: Revised for JFY2017.
%============================================================
%=======================================
% form00_header.tex
%	General header for kakenhiLaTeX,  Moved over from form00_2010_header.tex.
%	2009-09-06 Taku Yamanaka (Osaka Univ.)
%==== General Version History ======================================
% 2006-05-30 Taku Yamanaka (Physics Dept., Osaka Univ.)
% 2006-06-02 V1.0
% 2006-06-14 V1.1 Use automatic calculation for cost tables.
% 2006-06-18 V1.2 Split user's contents and the format.
% 2006-06-20 V1.3 Reorganized user and format files
% 2006-06-25 V1.4 Readjusted all the table column widths with p{...}.
%				With \KLTabR and \KLTabRNum, now the items can be right-justified
%				in the cell defined by p{...}.
% 2006-06-26 V1.5 Use \newlength and \setlength, instead of \newcommand, to define positions.
% 2006-08-19 V1.6 Remade it for 2007 JFY version.
% 2006-09-05 V1.7 Added font declarations suggested by Hoshino@Meisei Univ.
% 2006-09-06 V1.8 Introduced usePDFform flag to switch the form file format.
% 2006-09-09 V1.9 Changed p.7, to allow different heights between years. (Thanks to Ytow.)
% 2006-09-11 V2.0 Added an option to show budget summary.
% 2006-09-13 V2.1 Added an option to show the group.
% 2006-09-14 V2.1.1 Cleaned up Kenkyush Chosho.
% 2006-09-21 V2.2 Generated under a new automatic development system.

% 2007-03-24 V3.0 Switched to a method using "picture" environment.

% 2007-08-14 V3.1 Switched to kakenhi3.sty.
% 2007-09-17 V3.2 Added \KLMaxYearCount
% 2008-03-08 V3.3 Remade it for 2009 JFY version\
% 2008-09-08 V3.4 Added \KLXf ... \KLXh.
% 2011-10-20 V5.0 Use kakenhi5.sty, to utilize array package in tabular environment.
% 2012-08-14 v5.1 Moved preamble and kakenhi5 into the current directory, instead of the parent directory.
% 2012-11-10 v6.0 Switched to kakenhi6.sty.
% 2015-08-26 v6.1 Added KLFirstPageIsLongPage flag.
% 2018-02-12 Taku: Commented out \DeclareFontShape ...
%=======================================
%============================================================
% preamble.tex
%
% Dummy section and subsection commands.
% With these, some editors (such as TeXShop, etc.) can jump to the (sub)sections.
\newcommand{\dummy}{dummy}% 
\renewcommand{\section}[1]{\renewcommand{\dummy}{#1}}
\renewcommand{\subsection}[1]{\renewcommand{\dummy}{#1}}

% Flag for switching form file format.......
\usepackage{ifthen}
\newboolean{usePDFform}
\newboolean{BudgetSummary}

\usepackage{forms/kakenhi6}

\pagestyle{empty}

% ===== Parameters for LaTeX =========================

% ===== Font declarations  ======================================
%%\DeclareFontShape{JT1}{mc}{m}{it}{<->ssub * mc/m/n}{}
%%\DeclareFontShape{JY1}{mc}{m}{it}{<->ssub * mc/m/n}{}

% ===== Parameters for KL (Kakenhi LaTeX) ========================
% general purpose temporary variables	-2007
\newcommand{\KLX}{}
\newcommand{\KLXa}{}
\newcommand{\KLXb}{}
\newcommand{\KLXc}{}
\newcommand{\KLXd}{}
\newcommand{\KLXe}{}
\newcommand{\KLXf}{}
\newcommand{\KLXg}{}
\newcommand{\KLXh}{}
\newcommand{\KLY}{}
\newcommand{\KLYa}{}
\newcommand{\KLYb}{}
\newcommand{\KLYc}{}
\newcommand{\KLYd}{}
\newcommand{\KLYe}{}
\newcommand{\KLYf}{}
\newcommand{\KLXR}{}
\newlength{\KLCella}
\newlength{\KLCellb}
\newlength{\KLCellc}
\newlength{\KLCelld}
\newlength{\KLCelle}
\newlength{\KLCellf}
\newlength{\KLCellg}
\newlength{\KLCellh}

% sub-page
\newlength{\KLSubPageX}
\newlength{\KLSubPageY}
\newlength{\KLspx}
\newlength{\KLspy}
\newcommand{\KLSubPageXmm}{}	% for \input(x,y){....} which uses a unit (mm)
\newcommand{\KLSubPageYmm}{}	% for \input(x,y){....} which uses a unit (mm)

% margins for parbox inside frames; in units of points
\newcounter{KLParboxSideMargin}
\newcounter{KLParboxTopMargin}
\newcounter{KLParboxBottomMargin}

% ===== standard counters ======================================
\newcounter{KLSubPageNo}	% sub-page counter
\newcounter{KLPageOffset}		% to generate sub-page number
\newcounter{KLMaxYearCount}	% # of years for the proposal

% ===== standard flags ============================
\newboolean{KLFirstPageIsLongPage}

% ===== initializations ============
\KLInitTypesettingPageSelection
\newcommand{\KLCLLang}{}	% language-dependent left-justification in tabular



% user01_header
%=== 様式のファイルの形式の指定 =================
%   PDFではなく、eps の様式を読み込む場合は、次の行の頭に「%」をつけてください。
\setboolean{usePDFform}{true}
%===================================
%==========================================================
% form01_header.tex
%	2014-03-02: Taku Yamanaka (Osaka Univ.)
%		This is called after usePDFform is set.
%		Originally, this part was in form07_header.tex, but then
%		\usepackage{color} that is called before it was not effective.
%		[dvipdfmx] is not used for eps forms, because it makes the forms
%		slightly larger than pdf forms.
%		
%==========================================================
% ===== File format for forms ===========================
\ifthenelse{\boolean{usePDFform}}{
	\newcommand{\KLFormFormat}{pdf}	\usepackage[dvipdfmx]{graphicx}
}{	\newcommand{\KLFormFormat}{eps}	\usepackage{graphicx}
}

%----------------------------------------------------------------------------


% user02_header
%=== 予算の表の印刷 =====================
% 予算の集計の表を出すためには、次の行の頭の%を消してください。
%\setboolean{BudgetSummary}{true}
%=================================

%=== For English, uncomment the next line to left-justify inside table columns.
%\renewcommand{\KLCLLang}{\KLCL}

% === 一部のページだけタイプセット ==============
% New in 2009 fall version!
% 選んだページだけタイプセットするには、次の例の頭の%を消し、並べてください。
% 複数のページを選ぶこともできます。
% 提出前には、必ず全てコメントアウト(頭に%をつける)してください。
%ーーーーーーーーーーーーーーーーーーーーーーーーーーーーーーーーー
%\KLTypesetPage{1}			% p.1 (or p.1を含む連続したページ),
%\KLTypesetPage{3}			% p.3 (or p.3を含む連続したページ),
%\KLTypesetPagesInRange{5}{6}	% p.5 ~ p.6,
%\KLTypesetPagesInRange{8}{10}	% and p.8 ~ p.10
%=================================

% ===== my favorite packages ====================================
% ここに、自分の使いたいパッケージを宣言して下さい。
\usepackage{wrapfig}
% \usepackage{amssymb}
%\usepackage{mb}
\usepackage[usenames]{color} % でも科研費の書類はグレースケールで印刷されます
\usepackage{udline}
\usepackage{amsmath}
%\DeclareGraphicsRule{.tif}{png}{.png}{`convert #1 `dirname #1`/`basename #1 .tif`.png}
%==========================================================

\newcommand{\KLShouKeiLine}[1]{\cline{#1}}
%もし、小計の上の線を取れと事務に言われたら、
%「そのようなことは、記入要項に書かれていないし、学振はそのようなことは気にしていない。」と
% 突っぱねる。
% それでもなお消せと理不尽なことを言われたら、次の行の 最初の「%」を消す。	
%\renewcommand{\KLShouKeiLine}[1]{}

\newcommand{\KLBudgetTableFontSize}{small}	% 予算の表のフォントの大きさ: small, footnotesize
\newcommand{\KLFundsTableFontSize}{small}	%応募中、受入れ予定の研究費のフォントの大きさ:normalsize, small, footnotesize

% ===== my personal definitions ==================================
% ここに、自分のよく使う記号などを定義して下さい。
\setlength {\fboxrule}{1pt} % 箱の枠線の太さを変える
\setlength {\fboxsep}{3pt} % 箱枠内側の余白幅を変える
\definecolor{my_gray}{gray}{0.9}
\def\rem#1{ {\bf\textcolor{red}{($\clubsuit$ #1 $\clubsuit$)}}}

\let\oldenumerate\enumerate
\renewcommand{\enumerate}{
\oldenumerate
\setlength{\itemsep}{1.2pt}
\setlength{\parskip}{0pt}
\setlength{\parsep}{0pt}
}

% hook3: after including packages ===================
 % for future maintenance
% ===== Global definitions for the PD form ======================
% 基本情報
%
%------ 研究課題名  -------------------------------------------
\newcommand{\研究課題名}{粒子加速器を用いた電弱相互作用を持つ新物理の探索}

%----- 研究機関名と研究代表者の氏名-----------------------
\newcommand{\研究機関名}{東京大学}
\newcommand{\申請者氏名}{千草颯}
\newcommand{\研究代表者氏名}{\申請者氏名}

%---- 研究期間の最終年度 ----------------
\newcommand{\研究期間の最終元号年度}{34}	%平成で、半角数字のみ
%=========================================================
% ===== Global year-dependent definitions for the Kakenhi form ===========
% 基本情報
\newcommand{\研究開始年度}{2020}
\newcommand{\研究開始元号年度}{32}	%平成

\newcommand{\1年目西暦}{2020}
\newcommand{\2年目西暦}{2021}
\newcommand{\3年目西暦}{2022}
\newcommand{\4年目西暦}{2023}
\newcommand{\5年目西暦}{2024}
\newcommand{\6年目西暦}{2025}

\newcommand{\1年目}{32}
\newcommand{\2年目}{33}
\newcommand{\3年目}{34}
\newcommand{\4年目}{35}
\newcommand{\5年目}{36}
\newcommand{\6年目}{37}

\newcommand{\1年目J}{32}
\newcommand{\2年目J}{33}
\newcommand{\3年目J}{34}
\newcommand{\4年目J}{35}
\newcommand{\5年目J}{36}
\newcommand{\6年目J}{37}


	% 
%==========================================================
% form03_header.tex
%	2009-03-04: Taku Yamanaka (Osaka Univ.)
%==========================================================
\usepackage{calc}
\usepackage{watermark}
\usepackage{longtable}
\usepackage{geometry}                % See geometry.pdf to learn the layout options. There are lots.
\usepackage{udline}
\usepackage{array}

\geometry{noheadfoot,scale=1}  %scale=1 resets margins to 0
\setlength{\unitlength}{1pt}

% define variables for positions ==========================
% picture environment location, in  units of points
\newcommand{\KLOddPictureX}{}
\newcommand{\KLEvenPictureX}{}
\newcommand{\KLPictureY}{}
\newcommand{\KLOddPictureInWaterMarkX}{}
\newcommand{\KLEvenPictureInWaterMarkX}{}
\newcommand{\KLPictureInWaterMarkY}{}

\newlength{\KLoddsidemargin}
\newlength{\KLevensidemargin}
\newlength{\KLtopmargin}

\newcounter{KLCOddPictureInWaterMarkX}
\newcounter{KLCEvenPictureInWaterMarkX}
\newcounter{KLCPictureInWaterMarkY}
\newcounter{KLCOddPictureX}
\newcounter{KLCEvenPictureX}
\newcounter{KLCPictureY}

%------------------------------------------------------------

\newcommand{\KLLeftEdge}{}
\newcommand{\KLRightEdge}{}

% standard margins for text in frames
\setcounter{KLParboxSideMargin}{7}
\setcounter{KLParboxTopMargin}{12}
\setcounter{KLParboxBottomMargin}{5}

%-----------------------------------------------------------
\newcommand{\KLTwoHLines}{\hline\hline}



%=================================================================
% form05_pd_header.tex
%	for the 2007(H19) Japanese Fiscal Year
%	2006-10-01 : Taku Yamanaka (Osaka Univ.)
%			Switched to the new development system using a "mother file".
%	2007-08-08: Taku
%			Switched to a new method using "picture" environment.
%	2008-03-08: Taku
%			Readjusted parameters for the new 2008 form.
%	2009-09-04: Taku
%			Introduced form03_header and form07_header to automatically calculate margins and
%			other miscellaneous coordinate parameters.
%=================================================================

% ===== Global definitions for the Kakenhi form ======================
% 基本情報
\newcommand{\研究種目}{PD}
\newcommand{\研究種目後半}{}
\ifthenelse{\isundefined{\研究種別}}{
	\newcommand{\研究種別}{}
}{}%
\newcommand{\KLMainFile}{pd.tex}
\newcommand{\KLForms}{pd_forms}
\newcommand{\KLYoshiki}{pd}

% 奇数ページの下に記入される情報
\newcommand{\klbyYup}{}
\newcommand{\klbyYdown}{}
\newcommand{\klbyKikanXleft}{}
\newcommand{\klbyKikanXright}{}
\newcommand{\klbyNameXleft}{}
\newcommand{\klbyNameXright}{}

\newcommand{\KLBottomInfo}[6]{%
	\ifthenelse{\equal{#1}{}}{%
		\renewcommand{\klbyYup}{62}
		\renewcommand{\klbyYdown}{48}
	}{%
		\renewcommand{\klbyYup}{#1}
		\renewcommand{\klbyYdown}{#2}
	}
	
	\ifthenelse{\equal{#3}{}}{%
		\renewcommand{\klbyKikanXleft}{132}
		\renewcommand{\klbyKikanXright}{349}
		\renewcommand{\klbyNameXleft}{425}
		\renewcommand{\klbyNameXright}{550}
	}{%
		\renewcommand{\klbyKikanXleft}{#3}
		\renewcommand{\klbyKikanXright}{#4}
		\renewcommand{\klbyNameXleft}{#5}
		\renewcommand{\klbyNameXright}{#6}
	}
%	\KLTextBox{\klbyKikanXleft}{\klbyYup}{\klbyKikanXright}{\klbyYdown}{}{\研究機関名}%
	\KLTextBox{\klbyNameXleft}{\klbyYup}{\klbyNameXright}{\klbyYdown}{}{\申請者氏名}%
}

%==========================================================
% frame edge positions of multi-page-block
\newcommand{\KLOddMultiPageLeftEdge}{47}
\newcommand{\KLOddMultiPageRightEdge}{549}
\newcommand{\KLEvenMultiPageLeftEdge}{47}
\newcommand{\KLEvenMultiPageRightEdge}{549}

% vertical limits in the first multi-page-block
\newcommand{\KLMultiPageTopEdge}{806}		%lowest top position (except for the 1st page)
\newcommand{\KLMultiPageBottomEdge}{80}	%highest bottom position (except for the last page)

% Modify the edges for single page frames if necessary
\newcommand{\KLOddLeftEdge}{\KLOddMultiPageLeftEdge}
\newcommand{\KLOddRightEdge}{\KLOddMultiPageRightEdge}
\newcommand{\KLEvenLeftEdge}{\KLEvenMultiPageLeftEdge}
\newcommand{\KLEvenRightEdge}{\KLEvenMultiPageRightEdge}

%

%==========================================================
% form07_header.tex
%	2009-03-04: Taku Yamanaka (Osaka Univ.)
%	2014-03-02: Taku: Moved graphics part to form01_header.tex.
%	2015-08-26: Taku: Added a test for \KLFirstPageIsLongPage.
%==========================================================
% Remember Standard Positions that were set in form05_xxxx_header.tex
\let \KLStandardOddMultiPageLeftEdge = \KLOddMultiPageLeftEdge
\let \KLStandardOddMultiPageRightEdge = \KLOddMultiPageRightEdge
\let \KLStandardEvenMultiPageLeftEdge = \KLEvenMultiPageLeftEdge
\let \KLStandardEvenMultiPageRightEdge = \KLEvenMultiPageRightEdge

\let \KLStandardMultiPageTopEdge = \KLMultiPageTopEdge
\let \KLStandardMultiPageBottomEdge = \KLMultiPageBottomEdge

\let \KLStandardOddLeftEdge = \KLOddLeftEdge
\let \KLStandardOddRightEdge = \KLOddRightEdge
\let \KLStandardEvenLeftEdge = \KLEvenLeftEdge
\let \KLStandardEvenRightEdge = \KLEvenRightEdge

%------ This should be set before \begin{document} ------
\KLStandardLengths
\KLStandardPositions

\ifthenelse{\boolean{KLFirstPageIsLongPage}}{%
	\setlength{\textheight}{10000pt}%
}{%
}
%----------------------------------------------------------------------------


%============================================================
%endPrelude

\begin{document}
% hook5 : right after \begin{document} ==============
 % for future maintenance
%============================================================
%     User Inputs
%============================================================

%form: pd_form_03-04.tex ; user: pd_03-04_preparation_etc.tex
%========== PD =========
%===== p. 03-04 現在までの研究状況 =============
\section{現在までの研究状況}
%watermark: w03_past_pd
\newcommand{\現在までの研究状況}{%
%begin  現在までの研究状況===================

\vspace*{1mm}

\fcolorbox{black}{my_gray}{これまでの研究の背景、問題点、解決方策}

\vspace*{1mm}

\pmb{\textbf{[背景]}}2012年に世界最大の加速器施設であるLHCがHiggs粒子
を発見したことで、素粒子の標準模型により予言される粒子全ての存在が確認
された。一方で、標準模型の枠内では説明が出来ない現象は未だ多数残されて
おり、宇宙の大半を占める\pmb{\ul{暗黒物質の正体}}や、宇宙を構成するバ
リオンと反バリオンの数の非対称性、ニュートリノ質量の起源などがその例で
ある。\rem{こんなに例いる?}こうした状況を受けて、素粒子の加速器実験は
引き続き様々な角度から新物理の探索を続けているが、その中で最も重要な視
点の一つが、\pmb{\ul{電弱相互作用の詳細な理解}}である。

\vspace*{1mm}

電弱相互作用が重要である理由として、Higgsが電弱相互作用に付随する対称
性の破れを担う粒子であることが挙げられる。Higgsを実際に加速器で生成す
ることが可能になったため、実験によるHiggsの性質の検証を通じて、電弱相
互作用の理解を深めるとともに、新物理の兆候を見出すという試みが多くなさ
れている。また別の理由として、電弱相互作用を持つ新粒子が暗黒物質の有力
な候補になる例が多く存在することも挙げられる。この事実の背景には、新粒
子と標準模型粒子とが電弱相互作用を通じて相互作用すると仮定すると、現在
の宇宙の暗黒物質残存量が、標準模型粒子に比較的近い質量を持つ新粒子によ
り説明され、理論的に興味深い様々な新物理の模型に埋め込みやすい、という
事情が存在する。以上に挙げた理由から、電弱相互作用を詳細に理解すること
は、現在の実験状況や理論的根拠を鑑みて、素粒子物理の非常に重要な課題で
あると言える。

\vspace*{1mm}

\pmb{\textbf{【問題点】}}問題点は \pmb{\ul{電弱相互作用の理解が未だ不
十分}} なことである。これは、電弱相互作用が非常に弱い相互作用であるこ
と、また電弱対称性の破れのエネルギースケールが比較的高く、最後の構成要
素であるHiggsが最近発見されたばかりであることに起因する。

\vspace*{1mm}

\pmb{\textbf{【解決方策】}}% 上で述べたとおり、 電弱相互作用は非常に弱
% い相互作用であり、加速器実験の結果に与える効果は小さいことが期待される。
% そこで、信号に対する感度を上げるため、\ul{(1)新物理模型の特徴を用いて
% 実験手法を工夫する}ことを考える。
高いエネルギースケールにおける電弱対称性の破れやそれに関わる新物理のダ
イナミクスを明らかにする上で、そのエネルギーに到達可能な粒子加速器によ
る実験を用いることは重要である。特にここでは、\ul{(1a)特徴的な信号を用
いた新物理探索}、および\ul{(1b)標準模型の精密測定}といったアプローチで
この問題に取り組む。また、加速器実験のデータを元に\ul{(2)理論的な考察
から新物理の模型を制限する}ことも、電弱相互作用の理解を深める上で有用
である。次の項目で、(1a)(1b)(2)それぞれのアプローチについてより具体的
に説明する。

\vspace*{1mm}

\fcolorbox{black}{my_gray}{研究目的、方法}(詳細は「これまでの研究結果」も参照)

\vspace*{1mm}

全体の研究目的は、電弱相互作用を有する様々な素粒子模型に対して、粒子加
速器を用いてそれらを発見し、その性質を測定する、あるいは何らかの制限を
与えることである。申請者はこれまで、将来の運営が計画されている、LHCの7
倍のエネルギーを持つ\ul{100TeV 粒子加速器を用いた新物理の探索}、および
\ul{LHCによるHiggs発見が種々の素粒子模型に与える制限}に着目し、上記の
問題に取り組んだ。以下に、それぞれの研究の目的、および方法について具体
的に述べる。

\vspace*{1mm}

■\pmb{\ul{【A】100TeV 粒子加速器を用いた新物理の探索}}(解決方策の(1a)(1b)に対応)

\vspace*{1mm}

\pmb{\textbf{【背景】}}陽子を加速して衝突させる$100\,\mathrm{TeV}$粒
子加速器の実験では、強い相互作用による標準模型の背景事象が観測される事
象の圧倒的多数を占めることが予想される。この背景事象の中から新物理の効
果を抜き出すためには、実験手法を工夫する必要がある。

\vspace*{1mm}

\pmb{\textbf{【目的、方法】}}目的は、電弱相互作用を持つ新粒子の検出に
高い感度を持つ実験手法を提案し、その手法により新物理の構造がどの程度ま
で明らかになるかを調べることである。次に、ここで用いた二つの手法を説明
する。(a)電気的に中性でない長寿命粒子が存在すると、これが検出器内をあ
る程度の距離飛んでから崩壊することで、\pmb{消失飛跡}と呼ばれる特徴的な
信号を残す。これを用いることで、背景事象の数を減らし長寿命粒子への感度
を高められる。(b)電弱相互作用の電荷を持つ重い粒子を考えると、レプトン
対生成過程に影響を与えるが、そのイベント数の変化は終状態エネルギースケー
ルの関数として特徴的なピークを持つことが知られている。この関数形を用い
てフィッティングを行うことで、膨大な背景事象から新粒子の情報を抜き出す
ことが出来る。(a)(b)いずれの手法についても、粒子の衝突、崩壊、検出過程
をシミュレートし、得られたイベント数を元に統計処理を行うことで、新物理
の検出効率を調べる。

\vspace*{1mm}

■\pmb{\ul{【B】LHCによるHiggs発見が種々の素粒子模型に与える制限}}(解決方策の(2)に対応)

\vspace*{1mm}

\pmb{\textbf{【背景】}}LHCにおいてHiggsが発見され、その質量や結合の強
さなども高精度で測定されている。これらの性質は種々の素粒子模型に対する
低エネルギーでの境界条件に対応し、模型に制限を与える。

\vspace*{1mm}

\pmb{\textbf{【目的、方法】}}目的は実験で測定されたHiggsの性質との整合
性から、素粒子模型に制限を与えることである。特に、Higgs質量の測定値は、
我々の宇宙が標準模型の準安定真空に対応する可能性を示唆する。この
\pmb{電弱真空の安定性}を詳細に議論することで、模型への制限を得ることを
考える。具体的には、調べる模型に対してHiggsの自己相互作用定数に関する
くりこみ群方程式を解き、得られた解を用いて真空崩壊率を評価し、真空の寿
命と宇宙の年齢の大小を比較することで、模型の妥当性を議論する。

\vspace*{1mm}

\fcolorbox{black}{my_gray}{これまでの研究結果}

\vspace*{1mm}

以下これまでの研究成果をアプローチ毎にまとめる。[4.研究業
績-(1)-3,4,7,8]および[4.研究業績-(6)-2]の全ての研究が博士課程で行った
ものである。\ul{全ての研究で申請者が中心となって解析を行った}。

\vspace*{1mm}

■\pmb{\ul{【ia】消失飛跡を用いた新物理の探索}}(上記 \pmb{【A】} 手法(a)に対応)

超対称性模型に含まれ、暗黒物質の有力候補でもあるWinoの検出可能性を調べ
た。消失飛跡の信号を用いることで、標準模型の背景事象はほぼ存在せず、暗
黒物質を説明出来る質量のWinoまで検出が可能であることを明らかにした。ま
たWino飛跡の速度の情報を用いることで、Winoの質量測定が可能であることを
示した。さらにゲージーノと呼ばれる他の新粒子が崩壊してWinoを作る過程を
考えることで、他のゲージーノの質量を決定し、衝突イベントを完全に再構成
出来ることを示した[4-(1)-8]。

\vspace*{1mm}

■\pmb{\ul{【ib】新物理の信号の関数形を用いた背景事象との識別}}(上記 \pmb{【A】} 手法(b)に対応)

超対称性模型に含まれるHiggsinoを始めとして、暗黒物質の残存量を正しく説
明できる、電弱相互作用を持った新粒子の検出可能性を調べた。他の手法では
観測しづらい短寿命Higgsinoに対してこの手法が特に有効で、現状で最も良い
検出効率を与えることを示した[4-(1)-7]。また新物理のイベント数への寄与
が持つピークの位置や高さから、新物理の質量や電荷、スピンなどの性質を抜
き出せることを示した[4-(6)-2]。

\vspace*{1mm}

■\pmb{\ul{【ii】電弱真空の安定性から導かれる素粒子模型への制限}}(上記 \pmb{【B】} に対応)

標準模型と同様、一つのスカラー場が電弱相転移を引き起こす模型に対し、真
空崩壊率を従来より精密に計算するための定式化を与えた。標準模型の場合に
Higgs自己相互作用定数のくりこみ群方程式を解いて真空崩壊率を計算し、電
弱真空の準安定性を再導出した[4-(1)-3]。さらにHiggsと結合する新粒子が存
在する模型で同様の解析を行い、新粒子の質量と結合定数に対して制限を与え
た[4-(1)-4]。

\vspace*{1mm}

\fcolorbox{black}{my_gray}{特色と独創的な点}

\vspace*{1mm}

%end  現在までの研究状況 ====================
}

%form: pd_form_05-07.tex ; user: pd_05-07_purpose.tex
%========== PD =========
%===== p. 05-07 これからの研究計画 =============
\section{これからの研究計画}
%watermark: w02_purpose_pd
\subsection{研究の背景}
\newcommand{\研究の背景}{%
%begin  研究の背景===================

\vspace*{1mm}

\fcolorbox{black}{my_gray}{これからの研究計画の背景}

\vspace*{1mm}
%%

%%
\vspace*{1mm}

\fcolorbox{black}{my_gray}{問題点と解決すべき点、着想に至った経緯等}

\vspace*{1mm}
%%

%%

% \begin{thebibliography}{99}
%  \bibitem{teramura} 寺村輝夫、「ぼくは王様 - ぞうのたまごのたまごやき」.
% \end{thebibliography}

%end  研究の背景 ====================
}

\subsection{研究の特色・独創的な点}
\newcommand{\研究の特色と独創的な点}{%
%begin  研究の特色と独創的な点===================

\vspace*{1mm}

\fcolorbox{black}{my_gray}{特色、独創的な点}

\vspace*{1mm}
%%

%%
\vspace*{1mm}

\fcolorbox{black}{my_gray}{位置付け、意義}

\vspace*{1mm}
%%

%%

\fcolorbox{black}{my_gray}{インパクトおよび将来の見通し}

%end  研究の特色と独創的な点 ====================
}

\subsection{研究目的}
\newcommand{\研究目的}{%
%begin  研究目的===================

\vspace*{1mm}

\fcolorbox{black}{my_gray}{全体の研究目的}

\vspace*{1mm}
%%

%%
\vspace*{1mm}

\fcolorbox{black}{my_gray}{具体的な研究目的と方法、内容}

\vspace*{1mm}
%%

%%

%end  研究目的 ====================
}

%====================================
%form: pd_form_08.tex ; user: pd_08_plan.tex
%========== PD =========
%===== p. 08 年次計画 =============
\section{年次計画}
\subsection{年次計画}
\newcommand{\採用までの準備}{%
%begin  採用までの準備===================



%end  採用までの準備 ====================
}

\newcommand{\年次計画1年目}{%
%begin  年次計画1年目 (figureやtable使用可)===================



%end  年次計画1年目 (figureやtable使用可) ====================
}

\newcommand{\年次計画2年目}{%
%begin  年次計画2年目 (figureやtable使用可)===================



%end  年次計画2年目 (figureやtable使用可) ====================
}

\newcommand{\年次計画3年目}{%
%begin  年次計画3年目 (figureやtable使用可)===================



%end  年次計画3年目 (figureやtable使用可) ====================
}

%    }
%form: pd_form_09.tex ; user: pd_09_rights.tex
%========== PD =========
%===== p. 09 人権の保護及び法令等の遵守への対応 =============
\subsection{人権の保護及び法令等の遵守への対応}
\newcommand{\受け入れ研究室の選定理由}{%
%begin  受け入れ研究室の選定理由===================



%end  受け入れ研究室の選定理由 ====================
}

\newcommand{\人権の保護及び法令等の遵守への対応}{%
%begin  人権の保護及び法令等の遵守への対応 ===================

該当しない。

%end  人権の保護及び法令等の遵守への対応 ====================
}

%form: pd_form_10-11.tex ; user: pd_10-11_publications.tex
%========== PD =========
%===== p. 10-11 研究業績 =============
\section{研究業績}
%watermark: w14_pub_pd
% 2008-03-08 Taku
% 2009-03-04 K.S.
% 2010-05-06 Taku
% 2017-03-02 Taku: Added \KLCheckPageLimit and \KLAdvancePages.
\subsection{学術雑誌(紀要・論文集等も含む)に発表した論文及び著書}
\newcommand{\学術雑誌等に発表した論文または著書}{%
%begin  学術雑誌等に発表した論文または著書===================
	
	\begin{enumerate}
	 \item[](査読有り)%===========================
	 \item \ul{S.~Chigusa} and T.~Moroi,
	       ``Bottom-tau unification in a supersymmetric model with anomaly-mediation,''
	       Phys.\ Rev.\ D {\bf 94} (2016) no.3,  035016
	 \item \ul{S.~Chigusa} and T.~Moroi,
	       ``Bottom-Tau Unification in Supersymmetric SU(5) Models with Extra Matters,''
	       PTEP {\bf 2017} (2017) no.6,  063B05
	 \item \ul{S.~Chigusa}, T.~Moroi and Y.~Shoji,
	       ``State-of-the-Art Calculation of the Decay Rate of Electroweak Vacuum in the Standard Model,''
	       Phys.\ Rev.\ Lett.\  {\bf 119} (2017) no.21,  211801
	 \item \ul{S.~Chigusa}, T.~Moroi and Y.~Shoji,
	       ``Decay Rate of Electroweak Vacuum in the Standard Model and Beyond,''
	       Phys.\ Rev.\ D {\bf 97} (2018) no.11,  116012
	 \item \ul{S.~Chigusa} and K.~Nakayama,
	       ``Anomalous Discrete Flavor Symmetry and Domain Wall Problem,''
	       Phys.\ Lett.\ B {\bf 788} (2019) 249
	 \item \ul{S.~Chigusa}, S.~Kasuya and K.~Nakayama,
	       ``Flavon Stabilization in Models with Discrete Flavor Symmetry,''
	       Phys.\ Lett.\ B {\bf 788} (2019) 494
	 \item \ul{S.~Chigusa}, Y.~Ema and T.~Moroi,
	       ``Probing electroweakly interacting massive particles with Drell–Yan process at 100 TeV hadron colliders,''
	       Phys.\ Lett.\ B {\bf 789} (2019) 106
	 \item S. Asai, \ul{S. Chigusa}, T. Kaji, T. Moroi, M. Saito,
	       R. Sawada, J. Tanaka, K. Terashi and K. Uno,
	       ``Studying gaugino masses in supersymmetric model at future 100 TeV $pp$ collider,''
	       \rem{Biblio??}

		       

		\item[](査読なし)%=============================

		       なし
		       
	\end{enumerate}
%end  学術雑誌等に発表した論文または著書 ====================
}

\subsection{学術雑誌等又は商業誌における解説・総説}
\newcommand{\学術雑誌等または商業誌における解説や総説}{%
%begin  学術雑誌等または商業誌における解説や総説===================

なし

%end  学術雑誌等または商業誌における解説や総説===================
}

\subsection{国際会議における発表}
\newcommand{\国際会議における発表}{%
%begin  国際会議における発表===================

(口頭、査読有り)
	\begin{enumerate}
	 \item \ul{S. Chigusa}${}^\circ$ and T. Moroi, 
	       ``Bottom-Tau unification in Supersymmetric Model with Anomaly-Mediation'',
	       SUSY 2016,
	       The University of Melbourne (Australia),
	       2016年7月
	 \item \ul{S. Chigusa}${}^\circ$ and T. Moroi,
	       ``Bottom-Tau Unification in Supersymmetric Models'',
	       New Physics Forum,
	       IPMU,
	       2017年2月
	 \item \ul{S. Chigusa}${}^\omega$, T. Moroi and Y. Shoji,
	       ``Decay Rate of the Electroweak Vacuum in the Standard Model and Beyond'',
	       Planck 2018,
	       University of Bonn (Germany),
	       2018年5月
	 \item \ul{S. Chigusa}${}^\circ$ and K. Nakayama,
	       ``Flavon Stabilization in Models with Discrete Flavor Symmetry'',
	       KEK-PH 2018 winter
	       KEK,
	       2018年12月	      
	\end{enumerate}
(ポスター、査読有り)
        \begin{enumerate}
	 \setcounter{enumi}{4}
	 \item \ul{S. Chigusa}${}^\circ$ and T. Moroi,
	       ``Bottom Tau Unification in Supersymmetric Models'',
	       Les Houches Summer School 2017,
	       Les Houches School of Physics (France),
	       2017年7月
	 \item \ul{S. Chigusa}${}^\omega$, T. Moroi and Y. Shoji,
	       ``Decay Rate of the Electroweak Vacuum in the Standard Model and Beyond'',
	       Cargese 2018 International Summer School,
	       Scientific Institute of Cargese (France),
	       2018年7月
	 \item \ul{S. Chigusa}${}^\circ$, Y. Ema and T. Moroi
	       ``Probing Electroweakly Interacting Massive Particles with Precision Measurements at 100 TeV Hadron Colliders'',
	       Higgs as a Probe of New Physics 2019,
	       Osaka University,
	       2019年2月
	\end{enumerate}
%end  国際会議における発表 ====================
}

\subsection{国内学会・シンポジウムにおける発表}
\newcommand{\国内学会やシンポジウムにおける発表}{%
%begin  国内学会やシンポジウムにおける発表===================
(口頭、査読有り)
	\begin{enumerate}
	 \item \ul{S. Chigusa}${}^\circ$ and T. Moroi,
	       ``Bottom-Tau unification in Supersymmetric Model with Anomaly-Mediation'',
	       日本物理学会秋季大会,
	       宮崎大学,
	       2016年9月
	 \item \ul{S. Chigusa}${}^\omega$, T. Moroi and Y. Shoji,
	       ``Zero Mode Problem in the Calculation of Decay Rate of the SM Electroweak vacuum'',
	       日本物理学会秋季大会,
	       信州大学,
	       2018年9月
	\end{enumerate}
(ポスター、査読有り)
         \begin{enumerate}
	  \setcounter{enumi}{2}
	  \item \ul{S. Chigusa}${}^\circ$ and T. Moroi,
		``Bottom Tau Unification in Supersymmetric Models'',
		素粒子物理学の進展 2017,
		京都大学,
		2017年8月
	  \item \ul{S. Chigusa}${}^\circ$, Y. Ema and T. Moroi
		``Indirect Search of WIMP Dark Matter at Future 100 TeV Collider'',
		素粒子物理学の進展 2018,
		京都大学,
		2018年8月
	 \end{enumerate}
%end  国内学会やシンポジウムにおける発表 ====================
}

\subsection{特許等}
\newcommand{\特許等}{%
% %begin  特許等===================

なし

% %end  特許等 ====================
}

\subsection{その他の業績}
\newcommand{\その他の業績}{%
%begin  その他の業績===================

(受賞歴)
\begin{enumerate}
 \item 国際会議 ``Higgs as a Probe of New Physics
       2019'' において、the Best Poster Award 受賞(発
       表 4-(3)-7 に基づく)
\end{enumerate}

(arXiv 投稿済、査読中)
\begin{enumerate}
 \setcounter{enumi}{1}
 \item T. Abe, \ul{S. Chigusa}, Y. Ema and T. Moroi
       ``~~~~''
       \rem{Preprint??}
\end{enumerate}

(対外講演)
\begin{enumerate}
 \setcounter{enumi}{2}
 \item 名古屋大学(2018/10/16)、北海道大学(2019/1/11)、KEK
       (2019/4/9)で自身の研究内容に関するセミナーを行った。
\end{enumerate}

(その他)
\begin{enumerate}
 \setcounter{enumi}{3}
 \item 学術振興会特別研究員 DC1 に採択:2017年4月--2020年3月
 \item 東京大学数物フロンティア・リーディング大学院(FMSP)のコース生
       に採択:2015年10月--2020年3月
\end{enumerate}
%end  その他の業績 ====================
}

%===========================================================
% hook9 : right before \end{document} ============

%endUserFiles
% hook7 : right before including forms ============
 % for future maintenance

% pd_forms
%=======================================
\ifthenelse{\boolean{BudgetSummary}\OR\boolean{klTypesetPage0}}{
	%============================================================
%  Warning cover page
%============================================================

\begin{picture}(0,0)(\KLOddPictureX,\KLPictureY)
	\KLParbox{100}{700}{550}{600}{t}{
		\LARGE
		提出前に次の行を以下のようにコメントアウトし、\\
		コンパイルし直してください。\\
		\hspace{2cm}\%\textbackslash setboolean\{BudgetSummary\}\{true\}\\
		\hspace{2cm}\%\textbackslash KLTypesetPage\{..\}\\
		\hspace{2cm}\%\textbackslash KLTypesetPagesInRange\{..\}\{..\}\\
	}
	\西暦
	\KLParbox{100}{550}{500}{500}{t}{
		\begin{center}
			\LARGE 予算と研究組織のまとめ \\
			\Large \today
		\end{center}
	}

	\KLTextBox{100}{500}{550}{300}{}{
		\Large
		研究種目: \研究種目\研究種別\研究種目後半\\
		研究期間: \研究開始年度(H\研究開始元号年度) 〜 H\研究期間の最終元号年度\\
		研究課題名:「\研究課題名」\\
		研究代表者:\研究代表者氏名\\
		研究機関名:\研究機関名\\
	}
\end{picture}
\clearpage


}{}

\KLInputIfPageInRangeIsSelected{1}{2}{forms/pd_form_03-04}
\KLInputIfPageInRangeIsSelected{3}{5}{forms/pd_form_05-07}
\KLInputIfSelected{6}{forms/pd_form_08}
\KLInputIfSelected{7}{forms/pd_form_09}
\KLInputIfPageInRangeIsSelected{8}{9}{forms/pd_form_10-11}

%========================================


%endFormatFile

% hook9 : right before \end{document} ============
 % for future maintenance
\end{document}
