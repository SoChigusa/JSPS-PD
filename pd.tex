%\documentclass[11pt,a4paper,uplatex,twoside,dvipdfmx]{ujarticle} 	% for uplatex
\documentclass[11pt,a4paper,twoside,dvipdfmx]{jarticle}		% platex
%==== 科研費LaTeX =============================================
%	2020(H32)年度 PD
%============================================================
% 2008-03-08: Taku Yamanaka (JSPS Research Center for Science Systems / Osaka Univ.)
% 2008-03-10: Taku; Fixed a bug which was missing p.7.
% 2009-03-04: K.S.: Revised for JFY2010.
% 2010-03-04: Taku: Revised for JFY2011.
% 2011-03-20: Taku: Revised for JFY2012.
% 2012-02-25: Taku: Revised for JFY2013.
% 2013-03-14: Taku: Revised for JFY2014.
% 2014-03-02: Taku: Revised for JFY2015.
% 2015-02-23: Taku: Revised for JFY2016.
% 2016-02-26: Taku: Revised for JFY2017.
%============================================================
\input{forms/form00_header}
% user01_header
%=== 様式のファイルの形式の指定 =================
%   PDFではなく、eps の様式を読み込む場合は、次の行の頭に「%」をつけてください。
\setboolean{usePDFform}{true}
%===================================
\input{forms/form01_header}

% user02_header
%=== 予算の表の印刷 =====================
% 予算の集計の表を出すためには、次の行の頭の%を消してください。
%\setboolean{BudgetSummary}{true}
%=================================

%=== For English, uncomment the next line to left-justify inside table columns.
%\renewcommand{\KLCLLang}{\KLCL}

% === 一部のページだけタイプセット ==============
% New in 2009 fall version!
% 選んだページだけタイプセットするには、次の例の頭の%を消し、並べてください。
% 複数のページを選ぶこともできます。
% 提出前には、必ず全てコメントアウト(頭に%をつける)してください。
%ーーーーーーーーーーーーーーーーーーーーーーーーーーーーーーーーー
%\KLTypesetPage{1}			% p.1 (or p.1を含む連続したページ),
%\KLTypesetPage{3}			% p.3 (or p.3を含む連続したページ),
%\KLTypesetPagesInRange{5}{6}	% p.5 ~ p.6,
%\KLTypesetPagesInRange{8}{10}	% and p.8 ~ p.10
%=================================

% ===== my favorite packages ====================================
% ここに、自分の使いたいパッケージを宣言して下さい。
\usepackage{wrapfig}
% \usepackage{amssymb}
%\usepackage{mb}
\usepackage[usenames]{color} % でも科研費の書類はグレースケールで印刷されます
\usepackage{udline}
\usepackage{amsmath}
%\DeclareGraphicsRule{.tif}{png}{.png}{`convert #1 `dirname #1`/`basename #1 .tif`.png}
%==========================================================

\newcommand{\KLShouKeiLine}[1]{\cline{#1}}
%もし、小計の上の線を取れと事務に言われたら、
%「そのようなことは、記入要項に書かれていないし、学振はそのようなことは気にしていない。」と
% 突っぱねる。
% それでもなお消せと理不尽なことを言われたら、次の行の 最初の「%」を消す。	
%\renewcommand{\KLShouKeiLine}[1]{}

\newcommand{\KLBudgetTableFontSize}{small}	% 予算の表のフォントの大きさ: small, footnotesize
\newcommand{\KLFundsTableFontSize}{small}	%応募中、受入れ予定の研究費のフォントの大きさ:normalsize, small, footnotesize

% ===== my personal definitions ==================================
% ここに、自分のよく使う記号などを定義して下さい。
\setlength {\fboxrule}{1pt} % 箱の枠線の太さを変える
\setlength {\fboxsep}{3pt} % 箱枠内側の余白幅を変える
\definecolor{my_gray}{gray}{0.9}
\def\rem#1{ {\bf\textcolor{red}{($\clubsuit$ #1 $\clubsuit$)}}}

\let\oldenumerate\enumerate
\renewcommand{\enumerate}{
\oldenumerate
\setlength{\itemsep}{1.2pt}
\setlength{\parskip}{0pt}
\setlength{\parsep}{0pt}
}

\input{forms/hook3} % for future maintenance
% ===== Global definitions for the PD form ======================
% 基本情報
%
%------ 研究課題名  -------------------------------------------
\newcommand{\研究課題名}{粒子加速器を用いた電弱相互作用を持つ新物理の探索}

%----- 研究機関名と研究代表者の氏名-----------------------
\newcommand{\研究機関名}{東京大学}
\newcommand{\申請者氏名}{千草颯}
\newcommand{\研究代表者氏名}{\申請者氏名}

%---- 研究期間の最終年度 ----------------
\newcommand{\研究期間の最終元号年度}{34}	%平成で、半角数字のみ
%=========================================================
% ===== Global year-dependent definitions for the Kakenhi form ===========
% 基本情報
\newcommand{\研究開始年度}{2020}
\newcommand{\研究開始元号年度}{32}	%平成

\newcommand{\1年目西暦}{2020}
\newcommand{\2年目西暦}{2021}
\newcommand{\3年目西暦}{2022}
\newcommand{\4年目西暦}{2023}
\newcommand{\5年目西暦}{2024}
\newcommand{\6年目西暦}{2025}

\newcommand{\1年目}{32}
\newcommand{\2年目}{33}
\newcommand{\3年目}{34}
\newcommand{\4年目}{35}
\newcommand{\5年目}{36}
\newcommand{\6年目}{37}

\newcommand{\1年目J}{32}
\newcommand{\2年目J}{33}
\newcommand{\3年目J}{34}
\newcommand{\4年目J}{35}
\newcommand{\5年目J}{36}
\newcommand{\6年目J}{37}


	% 
\input{forms/form03_header}
%=================================================================
% form05_pd_header.tex
%	for the 2007(H19) Japanese Fiscal Year
%	2006-10-01 : Taku Yamanaka (Osaka Univ.)
%			Switched to the new development system using a "mother file".
%	2007-08-08: Taku
%			Switched to a new method using "picture" environment.
%	2008-03-08: Taku
%			Readjusted parameters for the new 2008 form.
%	2009-09-04: Taku
%			Introduced form03_header and form07_header to automatically calculate margins and
%			other miscellaneous coordinate parameters.
%=================================================================

% ===== Global definitions for the Kakenhi form ======================
% 基本情報
\newcommand{\研究種目}{PD}
\newcommand{\研究種目後半}{}
\ifthenelse{\isundefined{\研究種別}}{
	\newcommand{\研究種別}{}
}{}%
\newcommand{\KLMainFile}{pd.tex}
\newcommand{\KLForms}{pd_forms}
\newcommand{\KLYoshiki}{pd}

% 奇数ページの下に記入される情報
\newcommand{\klbyYup}{}
\newcommand{\klbyYdown}{}
\newcommand{\klbyKikanXleft}{}
\newcommand{\klbyKikanXright}{}
\newcommand{\klbyNameXleft}{}
\newcommand{\klbyNameXright}{}

\newcommand{\KLBottomInfo}[6]{%
	\ifthenelse{\equal{#1}{}}{%
		\renewcommand{\klbyYup}{62}
		\renewcommand{\klbyYdown}{48}
	}{%
		\renewcommand{\klbyYup}{#1}
		\renewcommand{\klbyYdown}{#2}
	}
	
	\ifthenelse{\equal{#3}{}}{%
		\renewcommand{\klbyKikanXleft}{132}
		\renewcommand{\klbyKikanXright}{349}
		\renewcommand{\klbyNameXleft}{425}
		\renewcommand{\klbyNameXright}{550}
	}{%
		\renewcommand{\klbyKikanXleft}{#3}
		\renewcommand{\klbyKikanXright}{#4}
		\renewcommand{\klbyNameXleft}{#5}
		\renewcommand{\klbyNameXright}{#6}
	}
%	\KLTextBox{\klbyKikanXleft}{\klbyYup}{\klbyKikanXright}{\klbyYdown}{}{\研究機関名}%
	\KLTextBox{\klbyNameXleft}{\klbyYup}{\klbyNameXright}{\klbyYdown}{}{\申請者氏名}%
}

%==========================================================
% frame edge positions of multi-page-block
\newcommand{\KLOddMultiPageLeftEdge}{47}
\newcommand{\KLOddMultiPageRightEdge}{549}
\newcommand{\KLEvenMultiPageLeftEdge}{47}
\newcommand{\KLEvenMultiPageRightEdge}{549}

% vertical limits in the first multi-page-block
\newcommand{\KLMultiPageTopEdge}{806}		%lowest top position (except for the 1st page)
\newcommand{\KLMultiPageBottomEdge}{80}	%highest bottom position (except for the last page)

% Modify the edges for single page frames if necessary
\newcommand{\KLOddLeftEdge}{\KLOddMultiPageLeftEdge}
\newcommand{\KLOddRightEdge}{\KLOddMultiPageRightEdge}
\newcommand{\KLEvenLeftEdge}{\KLEvenMultiPageLeftEdge}
\newcommand{\KLEvenRightEdge}{\KLEvenMultiPageRightEdge}

%

\input{forms/form07_header}
%============================================================
%endPrelude

\begin{document}
\input{forms/hook5} % for future maintenance
%============================================================
%     User Inputs
%============================================================

%form: pd_form_03-04.tex ; user: pd_03-04_preparation_etc.tex
%========== PD =========
%===== p. 03-04 現在までの研究状況 =============
\section{現在までの研究状況}
%watermark: w03_past_pd
\newcommand{\現在までの研究状況}{%
%begin  現在までの研究状況===================

\vspace*{1mm}

\fcolorbox{black}{my_gray}{これまでの研究の背景、問題点、解決方策}

\vspace*{1mm}

\pmb{\textbf{[背景]}} 2012年に世界最大の加速器施設であるLHCがHiggs粒子
を発見したことで、素粒子の標準模型により予言される粒子全ての存在が確認
された。一方で、標準模型の枠内では説明が出来ない現象は未だ多数残されて
おり、宇宙の大半を占める\pmb{暗黒物質の正体}や、宇宙を構成するバリオン
と反バリオンの数の非対称性、ニュートリノ質量の起源などがその例である。
\rem{こんなに例いる?}\rem{真空崩壊}こうした状況を受けて、素粒子の加速
器実験は引き続き様々な角度から新物理の探索を続けているが、その中で最も
重要な視点の一つが、\pmb{電弱相互作用の詳細な理解}である。

\vspace*{1mm}

電弱相互作用が重要である理由として、Higgsが電弱相互作用に付随する対称
性の破れを担う粒子であることが挙げられる。Higgsを実際に加速器で生成す
ることが可能になったため、実験によるHiggsの性質の検証を通じて、電弱相
互作用の理解を深めるとともに、新物理の兆候を見出すという試みが多くなさ
れている。また別の理由として、電弱相互作用を持つ新粒子が暗黒物質の有力
な候補になる例が多く存在することも挙げられる。この事実の背景には、新粒
子と標準模型粒子とが電弱相互作用を通じて相互作用すると仮定すると、現在
の宇宙の暗黒物質残存量が、標準模型粒子に比較的近い質量を持つ新粒子によ
り説明され、理論的に興味深い様々な新物理の模型に埋め込みやすい、という
事情が存在する。以上に挙げた理由から、電弱相互作用を詳細に理解すること
は、現在の実験状況や理論的根拠を鑑みて、素粒子物理の非常に重要な課題で
あると言える。

\vspace*{1mm}

\pmb{\textbf{【問題点]}} 問題点は、電弱相互作用は非常に弱い相互作用で
あるため、多くの場合、標準模型のその他の過程から生じる背景事象の発生頻
度と比べて、観測したい過程の発生頻度が非常に低いことである。特にLHCの
ようなハドロン同士を衝突させる実験装置では、強い相互作用に起因する背景
事象が圧倒的多数を占め、そこから統計・系統誤差の影響を考慮しつつ目的の
信号の情報を抜き出すためには工夫が必要となる。

\vspace*{1mm}

\pmb{\textbf{【解決方策】}} 実験データの中から欲しい情報を抜き出すため
に、新物理の効果に特有の特徴的な振る舞いを用いることを考える。この工夫
により期待できる利点としては、\ul{(1)解析に回される背景事象の数を大幅
に減らす}ことができる、あるいは\ul{(2)適切な統計処理により新物理の効果
を選び出す}ことができる、などが考えられる。以下では、(1)(2)それぞれの
場合について、これまでの研究経過と絡めて説明する。

\vspace*{1mm}

\fcolorbox{black}{my_gray}{研究目的、方法}

\vspace*{1mm}

全体の研究目的は、電弱相互作用を有する様々な新物理の候補に対して、粒子
加速器を用いてそれらを発見し、またその性質を測定する手法を与えることで
ある。申請者は特に、将来の運営が計画されている、LHCの7倍のエネルギーを
持つ100TeV 粒子加速器を用いて新物理を探索する手法について研究してきた。
これまで問題点の解決方策として用いた手法には、\ul{消失飛跡を用いた長寿
命粒子の探索}、および\ul{新物理の信号の関数形を用いた背景事象との識別}
が挙げられる。以下それぞれについて具体的に述べる。

\vspace*{1mm}

■\pmb{\ul{【A】消失飛跡を用いた長寿命粒子の探索}}(解決方策の(1)に対応)

\rem{背景として、AMSBならWinoが最後に残ることに触れるか?}

\pmb{【目的、方法】} 目的は、超対称性模型に含まれ、暗黒物質の有力候補
でもあるWinoを始めとして、ゲージーノと呼ばれる新粒子を探索し、その質量
を測定することである。特に、電磁力の電荷を持ったWinoは比較的長寿命であ
り、これが検出器内を一定距離飛んだ後で崩壊することにより、荷電粒子の飛
跡が途中で消失するような特徴的な信号を残すことを用いて、背景事象の数を
減らすことを考える。具体的には、粒子衝突でゲージーノが生成される過程、
およびその崩壊・検出過程をシミュレートし、得られたイベント数を標準模型
から期待される背景事象の数と比較することで、この手法によるゲージーノの
検出能力を調べる。また、Wino飛跡の速度の情報を用いることで、検出された
ゲージーノの質量を決定することを試みる。

■\pmb{\ul{【B】新物理の信号の関数形を用いた背景事象との識別}}(解決方策の(2)に対応)

\rem{こちらの背景は、Higgsinoが見つけづらいということ?}

\pmb{【目的、方法】} 目的は、超対称性模型に含まれるHiggsinoを始めとし
て、暗黒物質の残存量を正しく説明できる、電弱相互作用を持った新粒子を一
般に探索することである。ここでは、粒子加速器で最も頻繁に発生する過程の
一つであるレプトン対生成に着目し、それに対する量子補正の形で新物理が寄
与する効果を考える。特に、この効果の大きさを終状態レプトンのエネルギー
スケールの関数としてプロットしたとき、新粒子の質量の二倍の位置で急に折
れ曲がる特徴的なピーク構造が存在することが知られており、この構造を用い
て背景事象から新物理の効果を抽出することを目指す。具体的には、レプトン
対生成の過程をシミュレートすることで、標準模型および標準模型に新物理の
効果を入れた模型の二種類に対応するイベントを用意し、それらの間に統計的
に有意なズレが見られるかどうかを調べる。

\vspace*{1mm}

\fcolorbox{black}{my_gray}{これまでの研究結果}

\vspace*{1mm}

以下これまでの研究成果を【A】【B】それぞれのアプローチ毎に分けてまとめ
る。以下に述べる[4.研究業績-(1)-7,8]および[4.研究業績-(6)-2]の全ての研
究が博士課程で行ったものである。\ul{全ての研究で申請者が中心となって解
析を行った}。

\vspace*{1mm}

\fcolorbox{black}{my_gray}{特色と独創的な点}

\vspace*{1mm}

%end  現在までの研究状況 ====================
}

%form: pd_form_05-07.tex ; user: pd_05-07_purpose.tex
%========== PD =========
%===== p. 05-07 これからの研究計画 =============
\section{これからの研究計画}
%watermark: w02_purpose_pd
\subsection{研究の背景}
\newcommand{\研究の背景}{%
%begin  研究の背景===================

\vspace*{1mm}

\fcolorbox{black}{my_gray}{これからの研究計画の背景}

\vspace*{1mm}
%%

%%
\vspace*{1mm}

\fcolorbox{black}{my_gray}{問題点と解決すべき点、着想に至った経緯等}

\vspace*{1mm}
%%

%%

% \begin{thebibliography}{99}
%  \bibitem{teramura} 寺村輝夫、「ぼくは王様 - ぞうのたまごのたまごやき」.
% \end{thebibliography}

%end  研究の背景 ====================
}

\subsection{研究の特色・独創的な点}
\newcommand{\研究の特色と独創的な点}{%
%begin  研究の特色と独創的な点===================

\vspace*{1mm}

\fcolorbox{black}{my_gray}{特色、独創的な点}

\vspace*{1mm}
%%

%%
\vspace*{1mm}

\fcolorbox{black}{my_gray}{位置付け、意義}

\vspace*{1mm}
%%

%%

\fcolorbox{black}{my_gray}{インパクトおよび将来の見通し}

%end  研究の特色と独創的な点 ====================
}

\subsection{研究目的}
\newcommand{\研究目的}{%
%begin  研究目的===================

\vspace*{1mm}

\fcolorbox{black}{my_gray}{全体の研究目的}

\vspace*{1mm}
%%

%%
\vspace*{1mm}

\fcolorbox{black}{my_gray}{具体的な研究目的と方法、内容}

\vspace*{1mm}
%%

%%

%end  研究目的 ====================
}

%====================================
%form: pd_form_08.tex ; user: pd_08_plan.tex
%========== PD =========
%===== p. 08 年次計画 =============
\section{年次計画}
\subsection{年次計画}
\newcommand{\採用までの準備}{%
%begin  採用までの準備===================



%end  採用までの準備 ====================
}

\newcommand{\年次計画1年目}{%
%begin  年次計画1年目 (figureやtable使用可)===================



%end  年次計画1年目 (figureやtable使用可) ====================
}

\newcommand{\年次計画2年目}{%
%begin  年次計画2年目 (figureやtable使用可)===================



%end  年次計画2年目 (figureやtable使用可) ====================
}

\newcommand{\年次計画3年目}{%
%begin  年次計画3年目 (figureやtable使用可)===================



%end  年次計画3年目 (figureやtable使用可) ====================
}

%    }
%form: pd_form_09.tex ; user: pd_09_rights.tex
%========== PD =========
%===== p. 09 人権の保護及び法令等の遵守への対応 =============
\subsection{人権の保護及び法令等の遵守への対応}
\newcommand{\受け入れ研究室の選定理由}{%
%begin  受け入れ研究室の選定理由===================



%end  受け入れ研究室の選定理由 ====================
}

\newcommand{\人権の保護及び法令等の遵守への対応}{%
%begin  人権の保護及び法令等の遵守への対応 ===================

該当しない。

%end  人権の保護及び法令等の遵守への対応 ====================
}

%form: pd_form_10-11.tex ; user: pd_10-11_publications.tex
%========== PD =========
%===== p. 10-11 研究業績 =============
\section{研究業績}
%watermark: w14_pub_pd
% 2008-03-08 Taku
% 2009-03-04 K.S.
% 2010-05-06 Taku
% 2017-03-02 Taku: Added \KLCheckPageLimit and \KLAdvancePages.
\subsection{学術雑誌(紀要・論文集等も含む)に発表した論文及び著書}
\newcommand{\学術雑誌等に発表した論文または著書}{%
%begin  学術雑誌等に発表した論文または著書===================
	
	\begin{enumerate}
	 \item[](査読有り)%===========================
	 \item \ul{S.~Chigusa} and T.~Moroi,
	       ``Bottom-tau unification in a supersymmetric model with anomaly-mediation,''
	       Phys.\ Rev.\ D {\bf 94} (2016) no.3,  035016
	 \item \ul{S.~Chigusa} and T.~Moroi,
	       ``Bottom-Tau Unification in Supersymmetric SU(5) Models with Extra Matters,''
	       PTEP {\bf 2017} (2017) no.6,  063B05
	 \item \ul{S.~Chigusa}, T.~Moroi and Y.~Shoji,
	       ``State-of-the-Art Calculation of the Decay Rate of Electroweak Vacuum in the Standard Model,''
	       Phys.\ Rev.\ Lett.\  {\bf 119} (2017) no.21,  211801
	 \item \ul{S.~Chigusa}, T.~Moroi and Y.~Shoji,
	       ``Decay Rate of Electroweak Vacuum in the Standard Model and Beyond,''
	       Phys.\ Rev.\ D {\bf 97} (2018) no.11,  116012
	 \item \ul{S.~Chigusa} and K.~Nakayama,
	       ``Anomalous Discrete Flavor Symmetry and Domain Wall Problem,''
	       Phys.\ Lett.\ B {\bf 788} (2019) 249
	 \item \ul{S.~Chigusa}, S.~Kasuya and K.~Nakayama,
	       ``Flavon Stabilization in Models with Discrete Flavor Symmetry,''
	       Phys.\ Lett.\ B {\bf 788} (2019) 494
	 \item \ul{S.~Chigusa}, Y.~Ema and T.~Moroi,
	       ``Probing electroweakly interacting massive particles with Drell–Yan process at 100 TeV hadron colliders,''
	       Phys.\ Lett.\ B {\bf 789} (2019) 106
	 \item S. Asai, \ul{S. Chigusa}, T. Kaji, T. Moroi, M. Saito,
	       R. Sawada, J. Tanaka, K. Terashi and K. Uno,
	       ``Studying gaugino masses in supersymmetric model at future 100 TeV $pp$ collider,''
	       \rem{Biblio??}

		       

		\item[](査読なし)%=============================

		       なし
		       
	\end{enumerate}
%end  学術雑誌等に発表した論文または著書 ====================
}

\subsection{学術雑誌等又は商業誌における解説・総説}
\newcommand{\学術雑誌等または商業誌における解説や総説}{%
%begin  学術雑誌等または商業誌における解説や総説===================

なし

%end  学術雑誌等または商業誌における解説や総説===================
}

\subsection{国際会議における発表}
\newcommand{\国際会議における発表}{%
%begin  国際会議における発表===================

(口頭、査読有り)
	\begin{enumerate}
	 \item \ul{S. Chigusa}${}^\circ$ and T. Moroi, 
	       ``Bottom-Tau unification in Supersymmetric Model with Anomaly-Mediation'',
	       SUSY 2016,
	       The University of Melbourne (Australia),
	       2016年7月
	 \item \ul{S. Chigusa}${}^\circ$ and T. Moroi,
	       ``Bottom-Tau Unification in Supersymmetric Models'',
	       New Physics Forum,
	       IPMU,
	       2017年2月
	 \item \ul{S. Chigusa}${}^\omega$, T. Moroi and Y. Shoji,
	       ``Decay Rate of the Electroweak Vacuum in the Standard Model and Beyond'',
	       Planck 2018,
	       University of Bonn (Germany),
	       2018年5月
	 \item \ul{S. Chigusa}${}^\circ$ and K. Nakayama,
	       ``Flavon Stabilization in Models with Discrete Flavor Symmetry'',
	       KEK-PH 2018 winter
	       KEK,
	       2018年12月	      
	\end{enumerate}
(ポスター、査読有り)
        \begin{enumerate}
	 \setcounter{enumi}{4}
	 \item \ul{S. Chigusa}${}^\circ$ and T. Moroi,
	       ``Bottom Tau Unification in Supersymmetric Models'',
	       Les Houches Summer School 2017,
	       Les Houches School of Physics (France),
	       2017年7月
	 \item \ul{S. Chigusa}${}^\omega$, T. Moroi and Y. Shoji,
	       ``Decay Rate of the Electroweak Vacuum in the Standard Model and Beyond'',
	       Cargese 2018 International Summer School,
	       Scientific Institute of Cargese (France),
	       2018年7月
	 \item \ul{S. Chigusa}${}^\circ$, Y. Ema and T. Moroi
	       ``Probing Electroweakly Interacting Massive Particles with Precision Measurements at 100 TeV Hadron Colliders'',
	       Higgs as a Probe of New Physics 2019,
	       Osaka University,
	       2019年2月
	\end{enumerate}
%end  国際会議における発表 ====================
}

\subsection{国内学会・シンポジウムにおける発表}
\newcommand{\国内学会やシンポジウムにおける発表}{%
%begin  国内学会やシンポジウムにおける発表===================
(口頭、査読有り)
	\begin{enumerate}
	 \item \ul{S. Chigusa}${}^\circ$ and T. Moroi,
	       ``Bottom-Tau unification in Supersymmetric Model with Anomaly-Mediation'',
	       日本物理学会秋季大会,
	       宮崎大学,
	       2016年9月
	 \item \ul{S. Chigusa}${}^\omega$, T. Moroi and Y. Shoji,
	       ``Zero Mode Problem in the Calculation of Decay Rate of the SM Electroweak vacuum'',
	       日本物理学会秋季大会,
	       信州大学,
	       2018年9月
	\end{enumerate}
(ポスター、査読有り)
         \begin{enumerate}
	  \setcounter{enumi}{2}
	  \item \ul{S. Chigusa}${}^\circ$ and T. Moroi,
		``Bottom Tau Unification in Supersymmetric Models'',
		素粒子物理学の進展 2017,
		京都大学,
		2017年8月
	  \item \ul{S. Chigusa}${}^\circ$, Y. Ema and T. Moroi
		``Indirect Search of WIMP Dark Matter at Future 100 TeV Collider'',
		素粒子物理学の進展 2018,
		京都大学,
		2018年8月
	 \end{enumerate}
%end  国内学会やシンポジウムにおける発表 ====================
}

\subsection{特許等}
\newcommand{\特許等}{%
% %begin  特許等===================

なし

% %end  特許等 ====================
}

\subsection{その他の業績}
\newcommand{\その他の業績}{%
%begin  その他の業績===================

(受賞歴)
\begin{enumerate}
 \item 国際会議 ``Higgs as a Probe of New Physics
       2019'' において、the Best Poster Award 受賞(発
       表 4-(3)-7 に基づく)
\end{enumerate}

(arXiv 投稿済、査読中)
\begin{enumerate}
 \setcounter{enumi}{1}
 \item T. Abe, \ul{S. Chigusa}, Y. Ema and T. Moroi
       ``~~~~''
       \rem{Preprint??}
\end{enumerate}

(対外講演)
\begin{enumerate}
 \setcounter{enumi}{2}
 \item 名古屋大学(2018/10/16)、北海道大学(2019/1/11)、KEK
       (2019/4/9)で自身の研究内容に関するセミナーを行った。
\end{enumerate}

(その他)
\begin{enumerate}
 \setcounter{enumi}{3}
 \item 学術振興会特別研究員 DC1 に採択:2017年4月--2020年3月
 \item 東京大学数物フロンティア・リーディング大学院(FMSP)のコース生
       に採択:2015年10月--2020年3月
\end{enumerate}
%end  その他の業績 ====================
}

%===========================================================
% hook9 : right before \end{document} ============

%endUserFiles
\input{forms/hook7} % for future maintenance

% pd_forms
%=======================================
\ifthenelse{\boolean{BudgetSummary}\OR\boolean{klTypesetPage0}}{
	\input{forms/coverpage}
}{}

\KLInputIfPageInRangeIsSelected{1}{2}{forms/pd_form_03-04}
\KLInputIfPageInRangeIsSelected{3}{5}{forms/pd_form_05-07}
\KLInputIfSelected{6}{forms/pd_form_08}
\KLInputIfSelected{7}{forms/pd_form_09}
\KLInputIfPageInRangeIsSelected{8}{9}{forms/pd_form_10-11}

%========================================


%endFormatFile

\input{forms/hook9} % for future maintenance
\end{document}
